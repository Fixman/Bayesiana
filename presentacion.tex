\documentclass{beamer}

\mode<presentation>{%
	\usetheme{Warsaw}
}

\usepackage[spanish]{babel}
\usepackage[utf8]{inputenc}

\title{Intuitive Theories of Mind: \\ A Rational Approach to False Belief \\ \vspace{1ex} {\large Un Pequeño Análisis}}

\author[Grupo 9]{%
	Martín Fixman \and
	Axel Straminsky
}

\date{Primer Cuatrimestre 2016}

\begin{document}

\begin{frame}[fragile]
\titlepage{}
\end{frame}

\section{Introducción}

\begin{frame}[fragile]{Introducción}
Los niños pequeños suelen tener razonamientos falsos a ciertos problemas que parecen desconcentantes o dificiles de predecir para los adultos.

Esta presentación toma datos del paper de \textbf{Noah Goodman}\cite{goodman92} para demostrar que ambos modelos de razonamiento, tanto el de niños pequeños como el de niños mayores, pueden definirse como modelos Bayesianos de creencias falsas.
\end{frame}

\section{Experimento}

\subsection{Pregunta}

\begin{frame}[fragile]{El Problema}
\begin{block}{\ \vspace{1em}}
``Sally pone un juguete en una canasta, y luego va afuera a jugar. \\
Mientras ella está jugando, otra persona mueve el juguete a la caja. \\
Cuando Sally vuelve a dónde estaba su juguete, ¿a dónde lo va a buscar?''
\end{block}

\vspace{1em}

\alt<2>{%
En este experimento, se le hace esta pregunta a niños de diferentes edades y se categorizan sus respuestas.

Como es esperado, los niños de hasta 3 ó 4 años de edad responden que Sally va a buscar el juguete en la canasta, mientras que los mayores dicen que lo busca en la caja.
}{}
\end{frame}

\end{document}
